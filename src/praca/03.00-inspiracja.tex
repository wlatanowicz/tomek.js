\section{Źródła inspiracji}

Jednym z założeń projektowych było przygotowanie środowiska, które będzie ułatwiać pracę
programistom zajmującym się interfejsem użytkownika. Aby nie wprowadzać zupełnie nowych
rozwiązań, do których użytkownicy musieliby przywyknąć zdecydowano się na zaadoptowanie
już istniejących i mocno zakorzenionych na rynku rozwiązań.

Łatwość tworzenia aplikacji w środowisku Microsoft ASP.NET WebForms w znacznym stopniu miała
wpływ na decyzje co do ostatecznej implementacji tworzonego rozwiązania. Technika
\emph{Code-behind}, polegająca na oddzieleniu kodu warstwy aplikacji od logiki
biznesowej okazała się rozwiązaniem pożądanym podczas tworzenia projektu.
Wykorzystano również pomysł kompilacji szablonu warstwy prezentacji i klasy odpowiedzialnej
za logikę biznesową do jednego kodu wynikowego.

Bezpośrednim źródłem inspiracji, mającym wpływ na nazewnictwo komponentów oraz wygląd
szablonów warstwy prezentacji był framework PRADO. PRADO to napisany w PHP framework
bardzo zbliżony sposobem działania do rozwiązań oferowanych przez Microsoft. Niewątpliwą
zaletą PRADO jest otwartość kodu --- został on opublikowany na otwartej licencji BSD, co umożliwia
szczegółową analizę jego działania.