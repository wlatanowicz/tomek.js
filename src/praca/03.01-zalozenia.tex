\section{Założenia projektowe}

Aby powstające rozwiązanie było użyteczne i konkurencyjne w stosunku do rozwiązań
dostępnych na rynku przygotowano listę podstawowych założeń projektowych, które zostaną
omówione w niniejszej części.

\subsection{Szablony}

Podstawowym założeniem projektowym, implikującym użyteczność proponowanego framework'a
jest wykorzystanie szablonów. Programista ma możliwość tworzenia złożonych komponentów
poprzez składanie innych, prostszych. Aby ułatwić zadanie tworzenia szablonów zdecyowano,
że będą one definiowane w języku XML. Dzięki takiemu rozwiązaniu użytkownik nie musi
opanowywać nowej, skomplikowanej składni, jak ma to miejsce w niektórych dostępnych
na rynku rozwiązaniach np. Sencha (źródło !!!!!!!!).

Zaletą zastosowania języka XML do opisu szablonów jest fakt, że głównym językiem
stosowanym w aplikacjach internetowych jest HTML lub jedna z jego odmian. Fakt ten
w znaczący sposób ułatwia tworzenie tych części szablonów, które odpowiadają statycznym
fragmentom kodu HTML. Nie bez znaczenia pozostaje fakt, że zastosowanie tego języka
ułatwia również kompilację szablonów do kodu wynikowego.

Ponieważ szablon w procesie kompilacji podlega dekompozycji (nie wiem jak lepiej określić
XML traversing!!!!!!!!!!!), a następnie przygotowywane są na jego podstawie odpowiednie wywołania
funkcji tworzących dokument, nie ma możliwości, aby wynikowy kod HTML nie był kodem
poprawnym. Przekształcanie dokumentu XML do kodu JavaScript, tworzącego dokument
ma również znaczący wpływ na wydajność, dzięki pominięciu procesu parsowania tekstu
HTML w przeglądarce. Fakt ten udowodniono podczas badań opisanych w rozdziale ( tu numer !!!!!!!!!).

\subsection{Hierarchiczne komponenty}

Drugim istotnym założeniem projektowym jest fakt, że komponenty powinny tworzyć hierarchię.
Dzięki takiemu podejściu, kod odpowiedzialny za obsługę podstawowych funkcji komponentów
nie jest duplikowany. Ponadto bardziej złożone komponenty, znajdujące się niżej w hierarchii,
posiadają znacznie uproszczony kod. Łatwiejsze jest również testowanie tak zorganizowanego
środowiska.

Podstawowym problemem w implementacji hierarchicznych obiektów w języku JavaScript jest
specyficzne podejście do obsługi obiektowości w tym języku, znacznie różniące się od
tego znanego z języków takich jak C++, Java czy PHP. W języku JavaScript nie istnieje
bowiem pojęcie \emph{klasy}, a dziedziczenie odbywa się przez tzw. prototypowanie.

Ponieważ zdecydowano się na, w miarę możliwości, odtworzenie środowiska programistycznego
zbliżonego do tego znanego z WebForms czy PRADO, najlepszym rozwiązaniem wydawało się
zasymulowanie tzw. klasycznego dziedziczenia. Istnieje wiele rozwiązań, które, w mniej
lub bardziej udany sposób, wprowadzają do języka JavaScript pojęcie klasy i dziedziczenia.
W pracy zdecydowano się na wykorzystanie biblioteki \emph{Base.js} autorstwa Dean'a Edwards'a
(źródło !!!!!!!).

\subsection{Obsługa zdarzeń}

Ważnym aspektem, często traktowanym po macoszemu przez autorów framework'ów, jest wygodna
i efektywna obsługa zdarzeń. Zwykle obsługę zdarzeń należy podłączyć do komponentu
programistycznie, co wymaga od programisty dodatkowej pracy.

(!!!!!!!!)

\subsection{Mały i szybki kod wynikowy}

Większość obecnie stosowanych framework'ów JavaScript jest używana w środowiskach produkcyjnych
tzw. w wersji zminimalizowanej. Minimalizacja polega na usunięciu zbędnych białych znaków i,
niekiedy, refaktoryzacji nazw zmiennych w taki sposób, aby były krótsze. Idea minimalizacji
sprowadza się do zmniejszenia rozmiaru pliku, który przeglądarka musi pobrać i sparsować.

W proponowanym rozwiązaniu postanowiono pójść o krok dalej. Podczas kompilacji aplikacji
włączane w kod wynikowy są jedynie te komponenty, które zostały faktycznie użyte przez
programistę lub są wymagane przez inne komponenty. Takie podejście ma szczególny wpływ
na rozmiar małych, nieskomplikowanych aplikacji, w których większość zaawansowanych
funkcji framework'a nie jest wykorzystywana. Ponadto zysk będzie rósł w miarę rozwoju
framework'a o nowe komponenty.

W wielu rozwiązaniach szablony fragmentów HTML są przetwarzane bezpośrednio w przeglądarce,
tuż przed użyciem. Często są one zapisane w postaci kodu JSON lub HTML, co prowadzi
do konieczności dodatkowego parsowania.

W proponowanym podejściu szablon XML jest transformowany do kodu wynikowego JavaScript
już na etapie kompilacji. Dzięki temu nie ma potrzeby przetwarzania dodatkowych, skomplikowanych
struktur danych w przeglądarce, gdyż cały szablon został sprowadzony do uruchamialnego
kodu JavaScript.

\subsection{Kompatybilność}

\subsection{Brak zewnętrznych bibliotek}

\subsection{Zgodność z innymi bibliotekami}
